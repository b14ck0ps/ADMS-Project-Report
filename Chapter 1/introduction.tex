\chapter{Introduction}
\hrulefill

Esports Management System is an innovative platform that will revolutionize the management and organization of esports teams, participants, tournaments, and sponsors. This system seeks to provide users with an efficient and user-friendly way to search for their preferred professional esports players.

A user-friendly interface is at the core of the Esports Management System, allowing users to seamlessly navigate and explore the realm of professional esports. With only a few clicks, users can search for potential professional athletes and teams, as well as access valuable information such as their winning records and accomplishments. This enables fans and enthusiasts to remain up-to-date on their preferred players and teams, nurturing a stronger connection within the esports community.

The Esports Management System's ability to facilitate sponsorships is a crucial feature. Numerous organizations and businesses can engage in sponsorship activities, whether for the purpose of supporting tournaments or individual athletes. The system serves as a centralized repository where the information and details of these sponsors can be efficiently stored and managed. This facilitates the sponsorship process and ensures that sponsors and the esports industry collaborate effectively.

There are specialized administrators within the Esports Management System who play crucial roles in managing and enhancing the overall experience. The social media manager is among these supervisors; he or she supervises the organization's online presence and engagement on various social media platforms. In addition, the content creator/VFX/GFX team assures the creation of visually stunning and captivating content that enhances the overall esports experience.

Dynamic features and functionalities make the Esports Management System an indispensable instrument for the esports industry. It makes it easier for fans to discover and connect with professional esports players, allowing them to remain informed and engaged. It enhances collaboration between organizations and the esports community by providing a centralized platform for sponsorship management. In addition, the system enables administrators to enhance the organization's online presence and develop visually appealing content, ensuring that all stakeholders have an engaging experience.

In the following sections, we will delve deeper into the features, functionalities, and innovative aspects of the Esports Management System, demonstrating its potential to revolutionize the management and celebration of esports teams, players, tournaments, and sponsors.

\section{Project Proposal}
This proposal for the development and implementation of an Esports Management System is presented with pleasure. This revolutionary platform seeks to transform the management and organization of esports teams, players, tournaments, and sponsors. The Esports Management System will improve the user experience, encourage community engagement, and expedite operations within the esports industry by leveraging advanced technology and comprehensive functionalities.

\subsection{Purposes}

\begin{itemize}
    \item Create an intuitive web-based platform that serves as the central hub for esports administration, catering to the requirements of teams, players, tournament organizers, and sponsors.
    \item Implement a sophisticated matching algorithm to facilitate the search and discovery of favored professional esports players, thereby enhancing the fan experience and fostering esports community connections.
    \item Provide efficient sponsorship administration capabilities, enabling organizations and businesses to support tournaments or individual athletes through sponsorship activities.
    \item Enhance the organization's online presence by supervising social media platforms and having the content creator/VFX/GFX team produce visually spectacular and engaging content.
\end{itemize}


\subsection{Methodology}

\subsubsection*{System Development:}

\begin{itemize}
    \item Conduct exhaustive investigation on the necessary requirements and features of an effective Esports Management System.
    \item  Utilize industry-standard programming languages and technologies to create a scalable and secure web-based platform.
    \item Implement a user-friendly interface with intuitive navigation in order to provide a seamless and enjoyable user experience.
\end{itemize}

\subsubsection*{ Matching Algorithm}
\begin{itemize}
    \item Collaboration with data scientists and psychologists to create a matching algorithm based on personality traits, values, and beliefs.
    \item Integrate the matching algorithm into the system to recommend professional esports players compatible with the user's preferences.
\end{itemize}

\subsubsection*{Sponsorship Management}
\begin{itemize}
    \item Create an all-encompassing sponsorship management module to facilitate collaborations between organizations and the esports industry.
    \item  Provide a centralized repository for sponsor information to facilitate communication and sponsorship efficiency.
\end{itemize}

\subsubsection*{Online Presence Enhancement}
\begin{itemize}
    \item Appoint a social media manager to supervise the organization's online presence and interact with the esports community.
    \item Appoint a social media manager to supervise the organization's online presence and interact with the esports community.
\end{itemize}

\clearpage

\section{Project Scenario}
\hrulefill
\vspace*{12pt}

Imagine an eSports organization called "eSports FTW" that manages various teams and tournaments in the gaming industry. The organization is led by an Admin who oversees the operations. The Admin entity contains attributes such as email, picture, password, name, and a unique ID.

Under the Admin, there are multiple Managers responsible for different departments. Each Manager has attributes including hire date, picture, salary, email, name, department ID, and a unique ID. One Manager specifically manages the Finance department, ensuring financial stability and handling the accounts for the organization. The Finance entity consists of attributes such as a unique ID, account number, and balance.

In addition to managing the finances, each Manager is in charge of a specific team. The Teams entity includes details such as the team's established date, country, name, team ID, team icon, winning numbers, and total prize money. Each team has a unique Manager assigned to it, ensuring proper coordination and organization. The Manager entity is linked to the Teams entity through the Manager ID attribute.

Within each team, there are multiple Players who represent the organization in various games. The Players entity contains attributes such as name, player ID, picture, salary, winning prize money, total hours played, phone number, and address (including country, city, zip code, and road number). Additionally, players have links to their social media profiles on platforms like Facebook, Instagram, Twitter, and YouTube.

The organization hosts tournaments, bringing together teams from different games. The Tournament entity consists of attributes such as tournament ID, name, prize pool, starting date, and ending date. Each tournament features various games such as Valorant, Mobile Legends: Bang Bang (MLBB), and Rainbow Six Siege. The Game entity contains attributes like name, release date, genre, game picture, publisher, platform, game ID, and prize pool. Each tournament may have different games associated with it, creating diverse competitive environments.

To support the teams and tournaments financially, eSports FTW seeks sponsorships from different companies. The Companies entity includes attributes such as name, company ID, location, sponsor date, and phone number. Multiple companies can sponsor both teams and tournaments, forming a many-to-many relationship between the Companies, Teams, and Tournament entities.

Additionally, eSports FTW employs a Social Media Manager responsible for managing the organization's online presence. The Social Media Manager entity contains attributes such as name, picture, email, manager ID, hire date, salary, phone number, and social media links (Facebook, Instagram, Twitter, YouTube). The Social Media Manager oversees the VFX/GFX and Content Creator teams, ensuring engaging content creation and visual effects. The VFX/GFX and Content Creator entities include attributes like name, picture, VFX/GFX ID, email, phone number, salary, and address (country, city, zip code, and road number).
